\documentclass[12pt]{article}

\usepackage[margin=1in]{geometry} 
\usepackage{amsmath,amsthm,amssymb,graphicx}
\usepackage{epstopdf}
\usepackage{xcolor}
\usepackage{arydshln}
\usepackage{listings}
\usepackage{accents}
\usepackage{bm}
%% commute diagram
\usepackage{tikz-cd}
\usepackage{amsmath}
%for graphs align
\usepackage{subcaption}
%% commute diagram
\usepackage{mathtools} 
\usepackage{comment}
%comment
\usepackage{todonotes}
\usepackage{bm}
\setlength{\marginparwidth}{2cm} %
%sort the citation
\usepackage[sort]{cite}

\usepackage{hyperref}
\usepackage{xcolor}
\hypersetup{
    colorlinks,
    linkcolor={red!50!black},
    citecolor={blue!50!black},
    urlcolor={blue!80!black}
}
\usepackage{cleveref} 

\newcommand{\pef}[1]{\todo[color=blue!30,inline]{{\bf PEF:} #1}}
\newcommand{\bda}[1]{\todo[color=green!30,inline]{{\bf BDA:} #1}}
\newcommand{\kh}[1]{\todo[color=brown!30,inline]{{\bf KH:} #1}}


\newcommand{\R}{\bm{R}}  
\newcommand{\Z}{\bm{Z}}
\newcommand{\N}{\bm{N}}
\newcommand{\Q}{\bm{Q}}

\theoremstyle{definition}
\newtheorem{definition}{Definition}[section]
\newtheorem{corollary}{Corollary}[section]
\newtheorem{question}{Question}[section]
\newtheorem{problem}{Problem}[section]
\newtheorem{theorem}{Theorem}[section]
\newtheorem{proposition}{Proposition}[section]
\newtheorem{example}{Example}[section]
\newtheorem{remark}{Remark}[section]
\newtheorem{lemma}[theorem]{Lemma}
\renewcommand\div{\operatorname{div}}
%\renewcommand\ker{\operatorname{ker}}
%\newcommand\range{\operatorname{range}}
\newcommand\coker{\mathrm{coker}}
\newcommand\divh{\operatorname{div_{\mathrm{h}}}}
\newcommand\curlh{\operatorname{curl_{\mathrm{h}}}}
\newcommand\gradh{\operatorname{grad_{\mathrm{h}}}}
\newcommand\curl{\operatorname{curl}}
\newcommand\grad{\operatorname{grad}}
\newcommand\rot{\operatorname{rot}}
\newcommand\Grad{\mathrm{Grad}}
\newcommand\Div{\mathrm{Div}}
\newcommand\dev{\mathrm{dev}}
\newcommand\Curl{\mathrm{Curl}}
\newcommand\Rey{\operatorname{Re}}
\newcommand\Rem{\operatorname{Re}_m}
\newcommand{\dx}{\,\mathrm{d}x}
\let\ker\undefined
\DeclareMathOperator{\ker}{ker}
\DeclareMathOperator{\range}{range}
\newenvironment{solution}{\begin{proof}[Solution]}{\end{proof}}
%color box
\usepackage{tcolorbox}
\newtcolorbox{myblock}[1]{colback=blue!5!white, colframe=blue!75!black, fonttitle=\bfseries, title=#1}

% review's comment
\newcommand{\rv}[1]{%
  \fcolorbox{black}{gray!20}{%
    \parbox{0.96\linewidth}{%
      \small #1
    }%
  }%
}

%response
\newcommand{\rp}{\textbf{\textcolor{blue}{Response:}}\newline}

%blue text for editing
\newcommand{\blue}[1]{\textcolor{blue}{#1}}

\begin{document}

% ------------------------------------------ %
%                 START HERE                  %
% ------------------------------------------ %

\title{Response to Reviewer's Comments on the manuscript: Topology-preserving discretization for
the magneto-frictional equations
arising in the Parker conjecture} % Replace with appropriate title
\author{Mingdong He, Patrick E.~Farrell, Kaibo Hu, Boris D.~Andrews} 

\maketitle
We sincerely thank the reviewer for the careful reading of our manuscript and for the thoughtful and constructive suggestions. We have done our best to address all comments in detail, and we have included the following files:
\begin{itemize}
    \item \textbf{Revised manuscript} with all updates incorporated.
    \item \textbf{diff.pdf} which highlights all changes made compared to the original submission.
    \item \textbf{Response letter} (this document) which provides a summary of major revisions and a point-by-point reply to the reviewer comments. 
\end{itemize}
Additionally, we made minor language improvements throughout the text to enhance readability; these changes are not color-marked. A summary of the main revisions and a point-by-point response to the reviewers, are provided below.
%\tableofcontents

\setlength{\parindent}{0pt}
\setlength{\parskip}{1em}
% -----------------------------------------------------
% The following two environments (theorem, proof) are
% where you will enter the statement and proof of your
% first problem for this assignment.
%
% In the theorem environment, you can replace the word
% "theorem" in the \begin and \end commands with
% "exercise", "problem", "lemma", etc., depending on
% what you are submitting. 
% -----------------------------------------------------
\section{Summary on Major Revisions}
\begin{enumerate}
    \item We have clarified the novelty of our work in the introduction, which has four aspects:
    \begin{enumerate}
        \item From our perspective, the broader contribution of this work (beyond the magneto-frictional equations) is to motivate \textbf{why structure-preserving schemes are needed}. In most settings, structure-preserving schemes are motivated by \emph{quantitative} improvements: non-structure-preserving schemes are fine, and structure-preserving ones are a bit better. For e.g.~the incompressible Stokes equations, using exactly divergence-free discretisations improves the error estimates, but standard schemes still give reasonable results. For the magneto-frictional equations, however, the improvement is \emph{qualitative}: the schemes that do not preserve helicity give highly unphysical trivial solutions.
        \item For the magneto-frictional equations, this is the first Eulerian scheme that preserves the helicity. Eulerian schemes are strongly desirable in computational solar physics, since they can avoid the mesh distortions introduced in Lagrangian methods \cite{longbottom1998magnetic, craig2005parker, wilmot2009magneticparallel, craig2014current,zhou2016formation, zhou2017constructing}. Such distortions can be so severe that they hinder further investigation of the Parker conjecture in astrophysics, e.g., for highly-twisted magnetic fields, the tangential discontinuities proposed by the Parker conjecture cannot be examined, even if the mesh is well resolved. 
    \item  Our scheme can achieve arbitrary order in time, while Hu, Lee, Xu [Ref.\ 23] mainly focused on a specific low-order time stepping scheme for a different (but closely related) problem. %When Hu wrote Ref.\ 23, we did not have a general framework for developing structure-preserving discretizations in time; the approach in that work was derived by heuristic means. Since writing Ref.\ 23 we now understand this much better, and one of the fruits of this is that we can now generalize to arbitrary order in time.
    \item Another novelty is our generalization of the notation of helicity and the Arnold inequality to domains with nontrivial topology. This covers the typical setting of the Parker conjecture.    
%    Another important novelty is our generalization of the notion of helicity and the Arnold inequality to the kind of domains that are the essential setting for the Parker conjecture.
    \end{enumerate}
    \item We have included more explanation of our scheme and how it connects with Ref.~23 and Ref.~2. 
    \item We have reworded the topology of field lines to helicity, to avoid the confusion between the topology of the domain and the topology of the field lines. 
\end{enumerate}

Please find below a point-to-point response to each of the reviewer's comments.

\section{Response to Reviewer 1}
\begin{enumerate}
    \item \rv{I expect the authors to add detailed references to
Ref.\ 23 whenever they adopt considerations, techniques, and notations from that paper,
whose lead author is even among the authors of the present paper.}

\rp 
We have added more details in Section 3.
%\rp Fixed. We apologize for the confusion and we have added more details for our schemes. For spatial discretisation, we indeed follow Ref.\ 23, and we regret that our manuscript did not make this clear. We assure the Reviewer that our omission was not deliberate, however; we certainly did not mean to throw red herrings. 
\begin{comment}
\pef{The text quoted here (and throughout the response) is out of sync with the main article. Please copy and paste \emph{all} text again once we have finalised our changes to the article.}
\end{comment}
This clarifies how our scheme connects with Ref.\ 23 and Ref.\ 2. 

    \item \rv{Title: What is “topology preserving”? The method is “helicity-preserving”! Also drop
“Parker conjecture” from the title, because this is hardly addressed in the paper.} 

\rp In this paper, ``topology-preservation'' means preserving the linkage and knotted structures of the magnetic field. In particular, helicity is a quantitative measure of knottedness. In the algorithms, helicity-preservation is achieved by using finite element de~Rham complexes, which preserves cohomology, another kind of topology. In the revised version, the title has been changed to \blue{\textit{Helicity-preserving finite element discretization for magnetic relaxation}} to make this distinction clear.

    \item \rv{“Contractible” is a concept from homotopy, but what matters in the context of the manuscript is co-homology. Therefore the authors should express all assumptions they
make on the topology of $\Omega$ in terms of Betti numbers.}
 
\rp We agree with the reviewer that ``contractible" is a concept from homotopy; a domain being contractible implies that all the cohomologies of degree greater than zero vanish. We have expressed our assumptions on the domain in terms of Betti numbers. 

\item \rv{Since $\bm B$ and $\bm H$ occur, maybe B should be called the “magnetic induction field” to distinguish it from the magnetic field $\bm H$}

\rp In MHD literature, there is a common and slight abuse of language since these two fields are often proportional due to the constitutive relation. We have revised Section 3 to clarify that $\bm H$ represents the auxiliary magnetic field. 

\item \rv{I think (3.5b) can only be derived using (3.2b). Please explain.}

\rp Fixed. Please see the new proof of Theorem 3.2.  

\item \rv{Give a reference for the fact that Gauss collocation Runge--Kutta methods preserve
quadratic invariants, of which the helicity is one specimen.}

\rp Fixed. Please see Section 3.1. 
\end{enumerate}



\section{Response to Reviewer 2}
\begin{enumerate}
    \item \rv{Therefore, I’m wondering whether the authors’ generalized
helicity can somehow be viewed a generalization of the $H^{BV}$ formula? That might give it 
a physical interpretation.}

\rp We thank the reviewer for pointing out the Bevir–Gray (BG) helicity as a possible physical interpretation. The BG construction addresses the same topological difficulty as our generalized helicity but uses a different mechanism—cutting surfaces—whereas our approach is based on harmonic forms. The definitions also differ in the role of the vector potential: in our setting, $\bm A$ is the pre-image of the curl-range part $\bm B_R$, while in BG, $\bm A$ appears to be given a priori with $\bm B = \curl \bm A$. A direct comparison is therefore not straightforward. We have revised Remark 3.10 to note these differences and indicate that exploring a precise correspondence is left for future work.


%\pef{Saying `We really appreciate this suggestion' makes it sound like we appreciate it more than those of the other reviewers, which is unwise. Reword}


%\pef{Is it exactly the same, or is it different? If it is different we should point it out. This is important because we claim this as one of the main novelties of our paper.}

\item \rv{Proof of Theorem 3.8: It seems to me that the last inequality is tacitly throwing away
what could be a significant fraction of the energy in practical applications where there is
a strong harmonic “guide field”. For example, refs [37-38] in the paper. If I understand
correctly, one could strengthen (3.15) to
\begin{equation}
    |\Tilde{\mathcal{H}}|\leq C^{-1} (\|\bm B\|^2 - \|\bm B_R\|^2)
\end{equation}
} 

\rp We could not see why (3.15) could be strengthened to 
\begin{equation}
    |\Tilde{\mathcal{H}}|\leq C^{-1} (\|\bm B\|^2 - \|\bm B_R\|^2)
\end{equation}
Since $\bm B =\bm B_R+\bm B_H$ is an $L^2$-orthogonal decomposition, the above inequality would imply that 
\begin{equation}
    |\Tilde{\mathcal{H}}|\leq C^{-1}\|\bm B_H\|^2,
\end{equation}
which does not seem to hold in general.

%Fixed. Thank you so much for pointing this out. If we can choose $\bm A$ with a proper gauge $(\bm A, \bm B_H)=0$, then we can strengthen the inequality as the author pointed out. 

%This is equivalent to saying that 
%\begin{equation}
 %   |\Tilde{\mathcal{H}}|\leq C^{-1} \|\bm B_R\|^2
%\end{equation}
%since $\|\bm B\|^2 = \|\bm B_R\|^2 +\|\bm B_H\|^2$ due to $\bm B_R \perp \bm B_H$. Please see Remark 3.9 of diff.pdf.

\item \rv{In other words, is this SP
finite-element method significantly more expensive than previous solvers?...So it would be good to know whether it would be
practical (in future work) for the resolution to be increased.}

\rp Our simulations were with 1,026,480 degrees of freedom. Our timestepping scheme is implicit, and so requires the solution of large nonlinear systems of equations at each timestep, and hence is more expensive than other approaches. For now we employ direct solvers for the arising linear systems, limiting our scalability. We have added a comment in the conclusions that in future work we will study the development of preconditioners to make our approach scalable to larger discretizations. Please see Section 5. 

\item \rv{I don’t know if this is
something that can/should be resolved through the anonymous referee process, but I feel
I should flag it up in case the installation instructions can be updated.} 

\rp Fixed. We apologize for this. The firedrake-zenodo process is less robust than we would like; it precisely records the software we used for the numerical results, but the firedrake installer sometimes fails to install older zenodo records after major updates. We have updated the install instructions and tested them on a new installation of Ubuntu. A new firedrake-zenodo has been generated. 

\item \rv{A related question. The use of avfet modules in the source code for making the figures
suggests to me that a “finite-element in time” discretisation is being used, whereas in L300
of the paper it says that the implicit midpoint method was used for the computations.
Could the authors clarify?}

\rp Many finite element in time schemes reduce to well-known time stepping schemes. %For example, continuous Lagrange elements of degree one in time reduces to an implicit midpoint rule; with the right choice of quadrature weights, higher-order versions of finite element in time schemes reduce to Gauss--Legendre schemes. 
In our case, we used continuous Lagrange elements of degree one in time, which reduces to an implicit midpoint rule. We added more explanations about this at the beginning of Section 4.

%\bda{It's a bit of work, but we could just switch the code over to use Irksome.}


%These are both true. Many finite element in time schemes reduce to familiar ones at lowest order (e.g.~using continuous Lagrange elements of degree 1 in time reduces to implicit midpoint; with the right choice of quadrature weights, higher order versions reduce to Gauss--Legendre schemes.) Please see Sec. 4 of diff.pdf. 
\end{enumerate}

We appreciate Reviewer 2's small suggestions, and we have addressed them all. These suggestions improve our paper and make it easier to read for a broader audience.

\begin{enumerate}
    \item \rv{for a more recent mathematical work on the Parker problem, consider citing “Obstructions to topological relaxation for generic magnetic fields”, A. Enciso and D. Peralta-Salas,
Arch. Rational Mech. Anal. 249, 6, 2025.}

\rp Fixed. We have now cited this paper in the introduction.

\item \rv{L25: suggest “fluid diffusion” →“fluid viscosity”}

\rp Fixed. 

\item \rv{Firstly, the domain of the integral should be specified (to make clear that it is a
volume integral).}

\rp Fixed. 

\item \rv{L48: here one could also cite “On the limitations of magneto-frictional relaxation”, A.
Yeates, Geophs. Astrophys. Fluid. Dyn. 116, 305, 2022 who gives a review of magneto-
friction in their introduction.}

\rp Fixed. We have now cited this paper in the introduction to the magneto-frictional equations.

\item \rv{L60: suggest to reword this sentence, because the magnetic energy is not non-increasing
in the original MHD system, only the total energy. (In particular, magnetic energy can be
converted to and from kinetic or internal energy.)}

\rp Fixed. Please see Eqn 1.5. 

\item \rv{Eq (1.4): Again, there is an assumption of boundary conditions here that would be better
clarified now rather than later.}

\rp Fixed. All the boundary conditions have been stated in advance. 

\item \rv{L70: “The existence of tangential discontinuities of the stationary solutions of (1.3) is
therefore equivalent to the Parker conjecture.” The Parker conjecture does not have
a universally-agreed statement, and this is only true for the “force-free” version of the
conjecture, as opposed to allowing more general magnetohydrostatic equilibria with gas
pressure gradients. This should probably be clarified.}

\rp Fixed. Now the statement of the Parker conjecture has been fixed to be the force-free version. Please see Section 1. 

\item \rv{L84: suggest to insert “method” after “mimetic Lagrangian”}

\rp Fixed. 

\item \rv{L113: spurious word “the”}

\rp Fixed.

\item \rv{L128: I think $\|\bm u\|_{0, 2} = \|\bm u\|$ should be the other way around?}

\rp Fixed.

\item \rv{Footnote 1 (p4), eq (3.1): I think there are too many curls on the right-hand side of the
first equation.}

\rp Fixed. We have removed the footnote and add details for explaining our scheme. Please see Section 3. 

\item \rv{L234-5: Probably my stupidity, but I don’t understand why it follows that $\bm B_H$ remains
constant?}

\rp Faraday's law
\begin{equation*}\label{faraday}
\partial_t \bm B=-\curl \bm E
\end{equation*}
implies that $\partial_t \bm B$ is in the range of $\curl$, which is orthogonal to the harmonic components. Therefore, $\bm B_H$ remains constant in the evolution. In other words, we consider a Hodge decomposition of \eqref{faraday}, and consider the time evolution of each component in the Hodge decomposition.

\item \rv{Eq (3.15): Is there a reason for not using $\mathcal{E}$ to denote the energy here, for consistency with
earlier?}

\rp Fixed. In the revised version, we used $\mathcal{E}$ for consistency. Please see Eqn 3.14

\item \rv{More explanation is required for the reader to understand the difference between
the “trivial” and “nontrivial” experimental setups.}

\rp Fixed. Please see Section 4. 

 
\item \rv{L302: Am I right that the magnitude of $\tau$ does nothing in this model other than rescale
the time variable? In which case, is there a particular reason for using 100 rather than 1?
As a side remark, note that in some implementations of magneto-friction in Solar Physics,
the $\tau$ parameter is made a function either of $\|\bm B\|_2$ or of space, which can have a significant
effect on the evolution.}

\rp Thank you for pointing this out. It is true that $\tau$ is related to the time scale and does not affect the helicity conservation. However, the choice of $\tau$ may affect the performance of numerical solvers. In this work, we fixed $\tau$ to be a constant number. In future work, we will address other choices of $\tau$, including adaptive time stepping. Please see Section 5. 

\bda{To be honest, I agree with this comment.

\quad The simulations use $\tau = 100$ and a timestep $\Delta t = 10$; using $\tau = 1$ and a timestep $\Delta t = 1000$ would yield exactly the same results \emph{(which, speaking of, is such a large timestep, no?)}.
Regardless, it doesn't really make sense to me to vary both.
Personally, my call would be just to remove $\tau$ altogether from the manuscript and say the simulations use $\Delta t = 1000$---the numerical results should be identical---however I understand that's a touch radical, so happy to hear other people's perspectives!

\quad N.B. I wouldn't say varying $\tau$ is adaptive timestepping either.
Sure it's \emph{equivalent} to adaptive timestepping, I would only really call it adaptive timestepping if you vary $\Delta t$.
Furthermore, I think varying $\Delta t$ would be much more robust: if I double $\tau$, $\bm u$ should double (making the solver take longer to converge); if I double $\Delta t$, it should not.}

\item \rv{L307: This sentence is extremely brief (for a non-specialist in finite elements) and would
benefit from at least a reference about how this projection is done.}

\rp Fixed. We have added more details of how we implement the projection. Please see Section 4.1. 

\item \rv{L320: Do we know the actual value of the Poincare constant C for simple rectangular
domains? This could give an idea of the tightness of the Arnol’d inequality in each case.
(Relates to my point 2 above.)}

\rp Thank you for pointing this out. At present, the calculation of the bound's tightness is beyond the scope of our current work. We plan to consider solving the eigenvalue problem to evaluate the bound's tightness for simple rectangular domains as well as more complex periodic domains in future research.


%In general computing the Poincar\'e constant involves solving an eigenproblem. However, it can be bounded in terms of the diameter $d$ of the domain: for bounded convex domains it is no more than $d/\pi$, and for nonconvex domains it is related to the diameter by a constant depending on how `non-convex' it is.

%\kh{The above comment is for the scalar Poincare inequality? Here we used a Poincare inequality for $H_0(\curl)$. We can still solve an eigenvalue problem. But do we have more explicit bounds? If not, we can perhaps add more details on the eigenproblem, either in response, or in the revised paper.}

\item \rv{Fig. 2: Can the authors explain why the helicity error increases at early times, before
becoming completely flat? (Not greatly important as these are very small errors.)}

\rp This tiny error might be the interpolation error as we first project the initial condition to a divergence-free space; We will consider this in the future if this error largely affects our numerical simulation. 
%\pef{We need a better answer here}
%\kh{Is this because of interpolation or projection of initial data? Did you use helicity of true initial data or discrete initial data in the plot?}
\end{enumerate}

\section{Response to Reviewer 3}

\begin{enumerate}
    \item \rv{ In my view, this manuscript is largely an extension of one of the author's previous works [23], which addresses a more complicated and more practical incompressible MHD system. The scheme (3.2) is largely identical to Algorithm 1 in [23]. The differences are twofold: (1) a much simpler equation for the velocity is considered in the current manuscript; (2) there is a type in (3.2c) as B in H(div) and its curl has to be computed weakly. The "projections" of auxiliary variables, which authors credited to [2], have been introduced in [23]. In addition, most properties in Section 3.1, if not all, have been proved in [23]. Therefore, I believe the novelty before Section 3.2 is very limited....The only novelty of this manuscript lies in Section 3.2, which extends Section 3.1 to a special non-contractible domain. One can argue the novelty lies in the introduction of a generalized helicity, i.e., Definition 3.5. The rest of the proof follows with the standard techniques from the authors' previous work such as [23]. This again shows the novelty is quite limited in the manuscript.}

\rp We apologize for the confusion and we have rewritten our introduction to clarify the novelty of our work. The novelties of our work in the introduction, which are as follows:
    \begin{enumerate}
        \item %We think the most important contribution of this work is to address the \textbf{fundamental question} of \textbf{why we need a structure-preserving scheme}. Even though there are many papers describing how to design a structure-preserving scheme, the question of why we need it is not addressed. Our work provides a striking example where to obtain a meaningful solution, it is crucial that we need to preserve the helicity on the discrete level.
        One of the major contributions of this work is to address the fundamental question of {\bf why a helicity-preserving algorithm is crucial for numerical computation}. Even though there have been several papers designing such methods, the role of helicity-preservation was sometimes not clearly motivated. This work demonstrates the huge difference between algorithms that preserve and do not preserve helicity, for magnetic relaxation problems.
        \item %This is the first Eulerian scheme that preserves the helicity for the magnetic relaxation problem, which is strongly desirable since it can avoid the mesh distortion compared to Lagrangian methods, which dominate in the astrophysics community. %We have thus addded the following sentence to our introduction.
        In solar physics, most codes for magnetic relaxation are based on Lagrangian methods, which have the disadvantage of mesh distortion. An Eulerian method is thus very desirable. To the best of our knowledge, this is the first work that solves the magnetic relaxation with a helicity-preserving Eulerian scheme.
        \item %The time discretisation of the scheme proposed in Ref.23 is restricted to low-order convergence in time; we did not have the unifying perspective we now possess. One substantial consequence of this is that our scheme can achieve arbitrary order in time, as well as space.  
        The time discretisation of the scheme   in Ref.~23 is restricted to low-order convergence in time; while in this work, the scheme can achieve arbitrary order in time. 

        \item % We further generalize the helicity and Arnold inequality to a nontrivial domain, which is the setting of the Parker conjecture. As our scheme works for both trivial and nontrivial domains, has attractive structure-preserving properties, and can achieve arbitrary order in time, and thus it is an ideal tool for astrophysics to investigate the Parker conjecture in the future. Our scheme also provides a good motivation for designing a discontinuity detector for the steady state $\bm B_{\infty}$ in the future, since this needs to build on the fact that we can get a correct steady state via correct numerical evolution. Otherwise, it is meaningless to detect the singularities of a wrong solution via an incorrect evolution. 
        We further generalize the helicity and Arnold inequality to a nontrivial domain, which is the setting of the Parker conjecture. As our scheme obtains qualitatively correct steady state, and works for both trivial and nontrivial domains, has desirable structure-preserving properties, and can achieve arbitrary order in time, it offers a promising tool for investigating the Parker conjecture in the future. 

%Our scheme also provides a good motivation for designing a discontinuity detector for the steady state $\bm B_{\infty}$ in the future, since this needs to build on the fact that we can get a correct steady state via correct numerical evolution. Otherwise, it is meaningless to detect the singularities of a wrong solution via an incorrect evolution. }
    \end{enumerate}

       We have fixed the typo in (3.1c) so that the curl now acts on the right variable.

\item \rv{The whole manuscript is written in a very unusual and somewhat confusing way. Both the abstract and introduction sounds like the manuscript directly addresses the Parker problem. However, there is no physically meaningful study performed by the authors for the Parker problem. The only numerical example is a case from [36], which however has nothing to do with the Parker problem. I found this is very puzzling, given that the authors are aware of the correct setting of the Parker problem like [33, Fig. 1] as pointed out on Page 6. However, they choose such a non-physical test case to demonstrate their scheme (further details below).}

\rp We apologize again for the confusion about the topology of the field lines and the topology of the domain. The test case from [36] was chosen as a benchmark with nontrivial helicity to illustrate the structure-preserving properties of our scheme. The test case from [36] is also an important physical example. We reword our expression so there is no conflation of the topology of the field lines and the topology of the domain. We have also added one more simulation called IsoHelix suggested in [36, Figure 1]. Both cases were widely tested for Lagrangian methods in the astrophysics community for investigation of the Parker conjecture. We think this will address the reviewer's concern that our test might be nonphysical. Please see Section 4.2 and Section 4.3.  

%\kh{I think we should add a test case as the reviewer suggested ([36, Figure 1]): periodic boundary on top and bottom; zero boundary on the side. I think Patrick had a plot of this in an earlier stage. Then we can first say that the Hopf test is interesting (citing papers); and we further included a test case related to the setting of the Parker problem. Topology is not relevant here. \\ Perhaps this is related to Gunnar's comment on zero helicity but nontrivial magnetic topology.}

\item \rv{To demonstrate the advantage of their scheme, the authors modify a case found in [36]. There are many issues in this modified test:  
First, this test was considered near the origin as a toroidal field in [36]. The authors claimed this is a "nontrivial" topology on Page 9. But the non-trivial field configuration is not the non-trivial topology (i.e., non-contractible domain) addressed by the manuscript. This is very confusing. Second, to show the advantage of their scheme, the authors apply a periodic boundary condition to this problem, which was not intended in [36]. This is obvious from (4.1) and Figures 6 and 7, that the B field decays so fast in the z direction and a periodic boundary condition makes little difference. Finally, to echo the previous point, the periodic boundary condition in this setting is meaningless as it is not a setup for the Parker problem where a dominant B field should be along the periodic direction.}

\rp  %In [36] (in the original reference), the field is described as having "nontrivial topology," referring to its structure near the origin. In our manuscript, "nontrivial topology" was intended to describe the non-contractible topology of the domain. To avoid confusion, we have revised the wording (e.g., to helicity when it refers to the topology of the field lines) and carefully reviewed the manuscript to clearly distinguish between the topology of the domain and that of the field.  
This is perhaps due to a confusion about the terminology. In the previous version, we used ``topology-preserving" to indicate either the topology (knots) of the magnetic fields, or the topology of the domain (cohomology and nontrivial harmonic fields). To make it clear, in the revised version, we changed to the term ``helicity-preserving" when we mean the knotted structures of the magnetic fields, and revised the article to clearly distinguish between the topology of the domain and that of the field.

In the modified test from [36], we imposed a periodic boundary condition. Although this is not the setup relevant to the Parker problem with a dominant $\bm B$ field along the periodic direction, the nontrivial {\it domain} topology does impose a challenge for numerical solvers, as a discrete harmonic field may exist in numerical schemes. To further address the setup relevant to the Parker problem, we included a new test called IsoHelix. Please see Section 4.2 and Section 4.3. 

\item \rv{I do not see the footnote 1 has anything to do with the paper, at least not with (3.2) as it claimed.}

\rp We have removed the footnote and added more explanation of how we arrived at our scheme. 

\item \rv{The trivial topology in the numerical section is never defined. I assume it is a case without periodic boundary condition. But providing more details is desired. Ironically, the trivial topology is a meaningful physical test considered in [36], while the non-trivial one is not.}

\rp Thank you for pointing this out. We have fixed such confusion. Please see Section 3.2 and Section 4. 

\item \rv{The paper is very short, at only 13 pages excluding the references. While a short paper is not an issue for a work of significant impact, much of its content appears to be incremental. As a result, I find the overall contribution of the paper to be quite limited.
}

\rp Again, for us, the most exciting aspect of this work is a \emph{compelling} physical example where structure preservation is \emph{crucial} for getting a reasonable numerical solution. We hope that this example will be of broader value to the SISC community.
\end{enumerate}

\clearpage
\bibliography{ref}
\bibliographystyle{siam}

\end{document}
